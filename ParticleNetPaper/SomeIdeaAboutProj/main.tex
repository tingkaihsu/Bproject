\documentclass[12pt]{article}
\usepackage{braket}
\usepackage{physics}
\usepackage{graphicx}
\usepackage{times}
\usepackage[export]{adjustbox}
\usepackage{listings}
\usepackage{mathcomp}
\usepackage{hyperref}
\usepackage{bm,amsmath}
\usepackage{float}
\usepackage{indentfirst}
\usepackage{bigints}
\numberwithin{equation}{section}

\title{Paper Reading(Only the introduction and conclusion)}
\author{Ting-Kai Hsu}
\date{2023.9.17}

\begin{document}
\maketitle
\tableofcontents

\section{Some Ideas}

In \href{https://arxiv.org/abs/2010.05464}{Boosted Higgs Boson Jet Reconstruction}, they point out a special technique \href{https://arxiv.org/abs/1903.09644}{\textit{jet grooming method}}, and it is a \textit{reinforcement learning} that can prevent \textit{plieup event}\footnote{Pileup event means that there are multiple events overlapping at the same time and same place, causing the model difficlut to distangle them.} during training.
This grooming method is designed for if there are too many particles from different origin clustered to form a jet, which is invalid; however, sometimes we would like to choose a larger range to contain the whole jet consituents and meet the pileup events.
However, the paper use another way to reconstruct the Higgs jet rather than the traditional way, that is, \textit{image based method}, they used the method called \textit{graph neural network}.
The GNN methods classify final state hadrons with particular relations to form jets, and the result is particlewise.
The GNN methods can encode the information of charge particle when processing as ParticleNet.
The GNN methods have been used in large applications, such as, \textit{pileup mitigation in HEP}, \textit{tracker\footnote{Tracker reconstruction refers to the process of reconstructing the paths or trajectories of charged particles as they pass through the detector.} reconstruction}
\href{https://arxiv.org/abs/2008.06064}{Ref} says that the GNN is capable of lableing constituents of a specific jet after supervised training.
The drawback of GNN is that it may take long time processing, which is very low-efficiencyed.
This paper adopts another method similar to GNN, that is, \textit{dynamic graph convolutional network}.
\\
\indent \textit{\textbf{Conculsion:} Representing each collider event as a point cloud, we adopt the GCN with focal loss to reconstruct the Higgs jet in the event.} Requiring the transverse momentum should be larger than $200\ \text{GeV}$, and then found out the high performance in Higgs jet tagging and momentum reconstruction.
GCN method is less sensitive to pileup contamination.
\\
\indent \textbf{Its performance is degraded when there are boosted particles other than the Higgs boson in an event.}
\\
\indent Following are some limitation of GCN:

\begin{itemize}
    \item The method can only reconstruct a single Higgs jet for each event (even in the event with multiple Higgs bosons).
    \item the current method is not efficient in detecting the Higgs jet in processes where the Higgs is accompanied by other energetic particles.
\end{itemize}


\end{document}